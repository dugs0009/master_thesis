% ----------------------------------------------------------------------%
% Coquille pour thèses et mémoires.                                   %
% UQAR, 29 mai 2009.                                                    %
% Modifié par F. cyr - aout 2013          								%
%		      AC. Tassel - janvier 2014                                 %
%		      C. Rigaud - janvier 2014                                  %
% ----------------------------------------------------------------------%

% ----------------------------------------------------------------------%
% 1- Préambule.                                                         %
% ----------------------------------------------------------------------%

% Pour le dépot initial (recto),       utiliser  oneside.
% Pour le dépot final   (recto-verso), remplacer oneside par twoside.

% \documentclass[12pt,oneside,francais,letterpaper]{stylethese}
\documentclass[12pt,twoside,letterpaper]{stylethese}

% \usepackage{multirow}
\usepackage{natbib}
\usepackage{amsmath}
\usepackage{enumerate}  	 % For fancy enumerate item labels.
% \usepackage{utopia}
\usepackage{txfonts,empheq}
% \usepackage{chancery}
%\usepackage{ccaption}		 % Pour utiliser \legend (pour l'explication de la figure,texte plus long)
%\usepackage{pgfplots}       % Pour dessiner des graphes direct dans LaTeX
%\pgfplotsset{compat=1.3}

% Sous Unix/Linux, utiliser  latin1 comme encodage.
% Sous Windows,    remplacer latin1 par ansinew. (??  vérifier, moi ??a marche comme ??a)
% Sous MacOS,      remplacer latin1 par applemac.


\usepackage{ae,lmodern} 	 % ou seulement l'un, ou l'autre, ou times etc.
\usepackage{babel}
\usepackage[utf8x]{inputenc}
\usepackage[T1]{fontenc}

%\usepackage{ucs}


\usepackage{chapterbib} % <----- for bibliography per chapter
%\usepackage[duplicate]{chapterbib}
%\usepackage{url}

% Interligne 1 et 1/2.
\setstretch{1.5}

\usepackage[linktocpage=true,linkcolor=cyan,citecolor=cyan,colorlinks=true]{hyperref}

% Pour cr??er un index (optionnel).

\makeindex

% Info sur la thèse (Titre, auteur, etc.)
\Titre{TITRE DE LA THÈSE} % Mettre en majuscule!! (je nai pas résolu ce problème...)
\Auteur{Sébastien Dugas}
\Faculte{programme de maîtrise en océanographie}
\Diplome{Master ès sciences}
\Date{Décembre}{2018}

% Utiliser \These pour une thèse, ou \Memoire pour un mémoire.
\These
% \Memoire

% Utiliser      \Chapitres pour une thèse (ou mèmoire) traditionnelle.
% Remplacer par \Articles  pour une thèse (ou mèmoire) par articles.
\Articles
% \Chapitres

\begin{document}
\pdfstringdefDisableCommands{%
\let\MakeUppercase\relax}
% ----------------------------------------------------------------------%
% 2- Liminaires de la thèse.                                           %
% ----------------------------------------------------------------------%
%----------------------------------------------------------------------%
% Liminaires de la thèse.                                              %
% UQAR septembre 2013                                                  %
% ---------------------------------------------------------------------%

% ----------------------------------------------------------------------%
% 1- Page titre.                                                        %
% ----------------------------------------------------------------------%

\Pagetitre
\cleardoublepage
% ----------------------------------------------------------------------%
% inclusions qui pourraient mériter d'être incluses dans le .cls 
% (commentez si non-nécessaire)
% 1.1 - Composition du Jury.                                           % 
\thispagestyle{empty}

\null
\vfill
\noindent \textbf{Composition du jury:}\\
\vspace{1cm}

\begin{singlespace}
  \noindent \textbf{[Prénom Nom], président du jury, [Université d’attache]}\\

  \noindent \textbf{Cédric Chavanne, directeur de recherche, Université du Québec à Rimouski}\\

  \noindent \textbf{Dany Dumont, codirecteur de recherche, Université du Québec à Rimouski}\\

  \noindent \textbf{[Prénom Nom], examinateur externe, [Université d’attache]}\\
\end{singlespace}

\vspace{2cm}
\noindent Dépôt initial le [date mois année] 
\hspace{3cm}
Dépôt final le [date mois année]


\cleardoublepage

% % 1.2 - Avertissement biblio.       
\thispagestyle{empty}

\vspace{2cm}
\begin{center}
UNIVERSITÉ DU QUÉBEC À RIMOUSKI\\
Service de la bibliothèque
\end{center}

\vspace{3cm}
\begin{center}
Avertissement
\end{center}


\vspace{1cm}

\noindent La diffusion de ce mémoire ou de cette thèse se fait dans le respect des droits de son auteur, qui a signé le formulaire {\itshape \og Autorisation de reproduire et de diffuser un rapport, un mémoire ou une thèse \fg}. 
En signant ce formulaire, l’auteur concède à l’Université du Québec à Rimouski une licence non exclusive d’utilisation et de publication de la totalité ou d’une partie importante de son travail de recherche pour des fins pédagogiques et non commerciales. 
Plus précisément, l’auteur autorise l’Université du Québec à Rimouski à reproduire, diffuser, prêter, distribuer ou vendre des copies de son travail de recherche à des fins non commerciales sur quelque support que ce soit, y compris l’Internet. 
Cette licence et cette autorisation n’entraînent pas une renonciation de la part de l’auteur à ses droits moraux ni à ses droits de propriété intellectuelle. 
Sauf entente contraire, l’auteur conserve la liberté de diffuser et de commercialiser ou non ce travail dont il possède un exemplaire.



\cleardoublepage  
% % 1.3 - Dedicace.   
\thispagestyle{empty}


\begin{minipage}[l]{0.45\textwidth}

\end{minipage}%
\hfill
\begin{minipage}[r]{0.5\textwidth}
\begin{quotation}
\begin{doublespace}

Voici ma dédicace... J'en met juste un peu plus long pour voir ce que ça donne...

\end{doublespace}
\end{quotation}
\end{minipage}%

\cleardoublepage       
% ----------------------------------------------------------------------%


% ----------------------------------------------------------------------%
% 2- Remerciements.                                                    %
% ----------------------------------------------------------------------%

\remerciements

\selectlanguage{english}

remerciement en anglais...

\selectlanguage{frenchb}

remerciements en francais...

% ----------------------------------------------------------------------%
% 3- Avant-propos.                                                     %
% ----------------------------------------------------------------------%

% \avantpropos

% ----------------------------------------------------------------------%
% 4- Resume/Abstract                                                           %
% ----------------------------------------------------------------------%

\resume
\begin{singlespace}    

  [Écrire le résumé ici]
  
  \begin{quote} 
    Mots clés: [Inscrire ici 5 à 10 mots clés]
  \end{quote}
\end{singlespace}
\cleardoublepage

\abstract
\begin{singlespace}    

  [Here's the abstract]    

  \begin{quote} 
    Keywords: [Inscrire ici 5 à 10 mots clés]
  \end{quote}
\end{singlespace}
\cleardoublepage




% ----------------------------------------------------------------------%
% 5- Table des matières.                                               %
% ----------------------------------------------------------------------%

\tabledesmatieres

% ----------------------------------------------------------------------%
% 6- Liste des tableaux.                                               %
% ----------------------------------------------------------------------%

\listedestableaux


% ----------------------------------------------------------------------%
% 7- Table des matières.                                               %
% ----------------------------------------------------------------------%

\listedesfigures


% ----------------------------------------------------------------------%
% 8- Liste des abréviations (optionnel).                               %
% ----------------------------------------------------------------------%

% \listeabrev
% \begin{liste}
% \item[SIGLE1] Ceci est la définition du premier sigle.

% \item[SIGLE2] Ceci est la définition du deuxième sigle.

% \item[SIGLE3] Ceci est la définition du troisième sigle.
% \end{liste}

% ----------------------------------------------------------------------%
% 9- Liste des symboles (optionnel).                                   %
% ----------------------------------------------------------------------%

% \listesymboles
% \begin{liste}
% \item[$\alpha$] Ceci est la définition du symbole $\alpha$ [unités].

% \item[$\omega$] Ceci est la définition du symbole $\omega$ [unités].

% \item[$\gamma$] Ceci est la définition du symbole $\gamma$ [unités].
% \end{liste}

% ----------------------------------------------------------------------%
% Fin des liminaires.                                                  %
% ----------------------------------------------------------------------%


\cleardoublepage


% ----------------------------------------------------------------------%
% 3- Corps de la thèse.                                                %
% ----------------------------------------------------------------------%

\debutcorps
\cleardoublepage
% Utiliser  <chapitres.tex> pour une thèse (ou mémoire) traditionnelle.
% Remplacer par articles.tex  pour une thèse (ou mémoire) par articles.


\introduction 
\selectlanguage{french}
\section*{Introduction générale - section 1}

Introdution générale à mettre dans cette section
 

\section*{Introduction générale - section 2}

Autre section si nécessaire
\cleardoublepage

This is a test \citep{Someauthor2013}

\selectlanguage{english}
\chapter{Titre du chapitre 1}


\section{abstract}

Here is the abstract

\section{Introduction}
\label{sec:intro}

Here an example of a citation \citep{Someauthor2013}

\section{Datasets and Methodology}
\label{sec:method}

etc.

\section{Conclusion}

etc.


%  ACKNOWLEDGMENTS

\section{acknowledgments}
This work was funded by bla bla bla


% Bibliography --------  Pour une biblio par chapitre!
% \begin{singlespace}
%   % \bibliographystyle{plainnat}
%   \bibliographystyle{elsarticle-harv} 
%   \bibliography{mylib}
% \end{singlespace}


%% Enter Figures and Tables here:

%ex:
% \begin{figure}
%   \noindent\includegraphics[width=39pc]{fig1.eps}
% \caption{}
% \label{f:fig1}
% \end{figure}


\chapter{Titre du chapitre 2}



\section{abstract}

Here is the abstract

\section{Introduction}
\label{sec:intro}

Here an example of a citation \citep{Someauthor2013}.

\section{Datasets and Methodology}
\label{sec:method}

etc.

\section{Conclusion}

etc.


%  ACKNOWLEDGMENTS

\section{acknowledgments}
This work was funded by bla bla bla


% Bibliography --------  Pour une biblio par chapitre!
% \begin{singlespace}
%   % \bibliographystyle{plainnat}
%   \bibliographystyle{elsarticle-harv} 
%   \bibliography{mylib}
% \end{singlespace}


%% Enter Figures and Tables here:

%ex:
% \begin{figure}
%   \noindent\includegraphics[width=39pc]{fig2.eps}
% \caption{}
% \label{f:fig2}
% \end{figure}


% \include{chap3}
\cleardoublepage

\conclusion
\selectlanguage{french}
\section*{Conclusion générale}


[insérer texte ici]
\cleardoublepage



% ----------------------------------------------------------------------%
% 4 - Bibliographie.                                                    %
% ----------------------------------------------------------------------%


\begin{singlespace}
  \makeatletter
  \phantomsection\addcontentsline{toc}{chapter}{\MakeUppercase{\@references}}
  \makeatother
  \selectlanguage{english}
  \bibdata{mylib}
  \bibliographystyle{elsarticle-harv} % Ici éditer le style
  \bibliography{mylib} % Ici mettre le nom de la biblio, ici mylib.bib
\end{singlespace}

% ----------------------------------------------------------------------%
% 5 - Appendices de la thèse.                                           %
% ----------------------------------------------------------------------%

%%% Local Variables: 
%%% mode: latex
%%% TeX-master: "these"
%%% End: 
\setstretch{1.5}
\selectlanguage{french}
\appendice{Titre de l'annexe 1}
\addtocounter{chapter}{1}
\setcounter{equation}{0}


Voici un exemple d'annexe pour le canevas \LaTeX des thèses et maitrises.


\section*{Exemple de section}


Je met ici une section pour voir ce que ça donne.



% ----------------------------------------------------------------------%
% Fin du document.                                                     %
% ----------------------------------------------------------------------%

\end{document}

